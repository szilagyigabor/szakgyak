\documentclass[10pt,a4paper,oneside]{report}
% nagyon sok kép esetén meggyorsítható a fordítás a draft móddal
% \documentclass[12pt,a4paper,oneside,draft]{report}
% ekkor a képek nem renderelődnek ki, csak placeholder lesz mérethelyesen
\usepackage[utf8]{inputenc} % mindenképp maradjon az utf-8 kódolás
\usepackage[magyar]{babel}
\usepackage[T1]{fontenc}
%\usepackage{amsmath}
%\usepackage{amsfonts}
%\usepackage{amssymb}
%\usepackage{graphics} % grafikus elemek, képek berakásához
%\usepackage{epsfig} % eps importáláshoz
%\usepackage{listings}
%\usepackage{sectsty}
%\usepackage{enumerate}
\usepackage{siunitx} % ezzel lehet hivatalosan megformázni: szám + mértékegység
%\usepackage{lastpage}
\usepackage{setspace}
\usepackage{hyperref} % PDF hivatkozásokhoz kell
%\usepackage[hang]{caption}
%\usepackage{titling} % a title, author parancsok szabad használatához
\usepackage{xcolor}
\usepackage{multicol}
\usepackage{blindtext}
\usepackage{wrapfig}


% A legendás siunitx package, ami a mértékegységek leírásának
% leghivatalosabb módja, ezeket a makrókat definiálja:

% használat: \SI[per-mode=symbol]{12345,678}{\kilo\amper\per\kandela\tothe{9}}
% ilyesmi lesz, csak jobb: 12345.67\, kAcd^{-9}

% egyszerűbb eset: \SI{50}{\ohm}


%% SI alapegységek:
% amper         -   \ampere
% kandela       -   \candela
% kelvin        -   \kelvin
% kilogramm     -   \kilogram
% méter         -   \metre  (!!)
% mól           -   \mole
% másodperc     -   \second

%% Származtatott SI egységek:
% becquerel     -   \becquerel
% celziusz fok  -   \degreeCelsius
% coulomb       -   \coulomb
% farád         -   \farad
% gray          -   \gray
% hertz         -   \hertz
% henry         -   \henry
% joule         -   \joule
% katal         -   \katal
% lumen         -   \lumen
% lux           -   \lux
% newton        -   \newton
% ohm           -   \ohm
% pascal        -   \pascal
% radián        -   \radian
% siemens       -   \siemens
% sievert       -   \sievert
% szteradián    -   \steradian
% tesla         -   \tesla
% volt          -   \volt
% watt          -   \watt
% weber         -   \weber

%% Elfogadott, nem SI mértékegységek
% nap           -   \day
% fok           -   \degree
% hektár        -   \hectare
% óra           -   \hour
% liter         -   \litre ( vagy \liter, de az előző kis "l" lesz, az utóbbi nagy "L")
% fokperc       -   \arcminute
% perc          -   \minute
% fokmásodperc  -   \arcsecond
% tonna         -   \tonne

%% Kísérleti úton megállapítható nem SI egységek:
% csillagászati egység      -   \astronomicalunit
% atomi tömegegység         -   \atomicmassunit
% bohr                      -   \bohr
% fénysebesség              -   \clight
% dalton                    -   \dalton
% elektrontömeg             -   \electronmass
% elektronvolt              -   \electronvolt
% egységtöltés              -   \elementarycharge
% hartree                   -   \hartree
% redukált planck állandó   -   \planckbar

%% Más nem SI egységek:
% ångström          -   \angstrom
% bár               -   \bar
% barn              -   \barn
% bel               -   \bel
% decibel           -   \decibel
% csomó             -   \knot
% higanymilliméter  -   \mmHg
% tengeri csomó     -   \nauticalmile
% neper             -   \neper

%% Prefixek:
% yocto-        -   \yocto  (-24)
% zepto-        -   \zepto  (-21)
% atto-         -   \atto   (-18)
% femto-        -   \femto  (-15)
% piko-         -   \pico   (-12)
% nano-         -   \nano   (-9)
% mikro-        -   \yocto  (-6)
% milli         -   \milli  (-3)
% centi-        -   \centi  (-2)
% deci-         -   \deci   (-1)

% deka-         -   \deca   (1)
% hekto-        -   \hecto  (2)
% kilo-         -   \kilo   (3)
% mega-         -   \mega   (6)
% giga-         -   \giga   (9)
% tera-         -   \tera   (12)
% peta-         -   \peta   (15)
% exa-          -   \exa    (18)
% zetta-        -   \zetta  (21)
% yotta-        -   \yotta  (24)


% a TikZ rajzoló modul, és a kapcsolási rajz készítő modul, ha kell
%\usepackage{tikz}
%\usepackage{circuitikz}

\pagenumbering{gobble}


\definecolor{rosewood}{rgb}{0.6, 0.0, 0.04}
\definecolor{indigo(dye)}{rgb}{0.0, 0.25, 0.42}

% az A4 oldal margóinak és méreteinek beállítása
\usepackage[left=15mm,right=15mm,top=10mm,bottom=5mm]{geometry}\pagestyle{plain}

% A sorköz távolság beállítása
% egyszeres sorköz
\singlespacing
% 1,5 sorköz
% \onehalfspacing

% A hivatkozások, és linkek átállítása alapértelmezett színre, fekete-fehér nyomtatáshoz optimalizálva
\hypersetup
{
  	colorlinks,
  	citecolor=blue,
 	linkcolor=rosewood,
  	urlcolor=indigo(dye)
}

\setlength{\columnsep}{1 cm}

\newcounter{magicrownumbers}
\newcommand\rownum{\stepcounter{magicrownumbers}\arabic{magicrownumbers}}


\begin{document}

\begin{center}
	\Large{Szakmai Gyakorlat Munkanapló}
\end{center}
\begin{tabular}{p{2.5 cm} p{2.5 cm} p{5 cm} p{6 cm}}
	\textbf{Név:} & Szilágyi Gábor & \textbf{Gyakorlat Időtartama:} & 11 hét\\
	\textbf{NEPTUN:} & NOMK01 & \textbf{Intézmény:} & Silicon Laboratories Hungary Kft.\\
	\textbf{Képzés:} & BSc & \textbf{Intézmény székhelye:} & 1033 Budapest, Ángel Sanz Briz út 13. \\
	& & \textbf{Gyakorlóhelyi Konzulens:} & Bódi Tamás
\end{tabular}
\begin{table}[h!]
	\centering
	\small
	\begin{tabular}{| c | p{3 cm} | p{10 cm} |}
	\hline
	 & Dátum & Tevékenység (Megjegyzés: Az egyes sorok 4 órát jelentenek) \\ \hline \hline
	\rownum & 2021.07.01. & IT oktatás, céges laptop beüzemelése \\ \hline
	\rownum & 2021.07.02. & Céges szoftverek telepítése \\ \hline
	\rownum & 2021.07.05. & Silabs EFR32 IC adatlap tanulmányozás \\ \hline
	\rownum & 2021.07.06. & Silabs EFR32 IC adatlap tanulmányozás \\ \hline
	\rownum & 2021.07.07. & New hire orientation call \\ \hline
	\rownum & 2021.07.09. & Zólomy Attila - Optikai Távközlés olvasása \\ \hline
	\rownum & 2021.07.09. & Zólomy Attila - Optikai Távközlés olvasása \\ \hline
	\rownum & 2021.07.13. & David M. Pozar - Microwave Engineering (Microwave Filters) \\ \hline
	\rownum & 2021.07.13. & Pozar - (Microwave Network Analysis) \\ \hline
	\rownum & 2021.07.14. & RAILtest demo program kipróbálása \\ \hline
	\rownum & 2021.07.15. & Pozar - (Impedance Matching and Tuning) \\ \hline
	\rownum & 2021.07.15. & Pozar - (Impedance Matching and Tuning) \\ \hline
	\rownum & 2021.07.19. & AN930.1 - 2.4 GHz matching guide olvasása \\ \hline
	\rownum & 2021.07.21. & AN923.1 - Sub-GHz matching guide olvasása \\ \hline
	\rownum & 2021.07.22. & AWR Microwave Office gyakorlás \\ \hline
	\rownum & 2021.07.22. & Pozar - (Microwave Filters) \\ \hline
	\rownum & 2021.07.23. & Pozar példaszűrő Microwave Office-ban \\ \hline
	\rownum & 2021.07.24. & Pozar - (Microwave Network Analysis) \\ \hline
	\rownum & 2021.07.25. & Pozar - (Transmission Lines and Waveguides) \\ \hline
	\rownum & 2021.07.25. & Pozar - (Transmission Lines and Waveguides) \\ \hline
	\rownum & 2021.08.06. & Vida Zoltán - Microwave Office gyorstalpaló\\ \hline
	\rownum & 2021.08.09. & Pozar példaszűrő MWO-ban, valós komponensekkel \\ \hline
	\rownum & 2021.08.10. & Pozar példaszűrő Microwave office-ban, yield-analízis \\ \hline
	\rownum & 2021.08.11. & Yield analízis debugolás \\ \hline
	\rownum & 2021.08.12. & Zólomy Attila - Optikai Távközlés olvasása \\ \hline
	\rownum & 2021.08.13. & Zólomy Attila - Optikai Távközlés olvasása \\ \hline
	\rownum & 2021.08.16. & Netlist importálás MWO-ba, yield analízis \\ \hline
	\rownum & 2021.08.17. & Kötelező tréningek elvégzése \\ \hline
	\rownum & 2021.08.17. & Kötelező tréningek elvégzése \\ \hline
	\rownum & 2021.08.18. & Kötelező tréningek elvégzése \\ \hline
	\rownum & 2021.08.19. & Irodalomkutatás: szűrő típusok, SAW-filter \\ \hline
	\rownum & 2021.08.24. & Hardware-es meeting \\ \hline
	\rownum & 2021.08.26. & Zólomy matching példák szimulációja MWO-ban és Smith-ben \\ \hline
	\rownum & 2021.08.26. & Zólomy matching példák szimulációja MWO-ban és Smith-ben \\ \hline
	\rownum & 2021.08.27. & Range test példaprogram kipróbálása \\ \hline
	\rownum & 2021.08.27. & Range számolás \\ \hline
	\rownum & 2021.08.30. & Range számolás \\ \hline
	\rownum & 2021.08.31. & Altium és Pads bevezető tréning \\ \hline
	\rownum & 2021.09.06. & Ismerkedés a Jenkins build rendszerrel \\ \hline
	\rownum & 2021.09.06. & Ismerkedés a Jenkins Build rendszerrel \\ \hline
	\rownum & 2021.09.07. & Layout tréning \\ \hline
	\rownum & 2021.09.08. & Céges kültéri esemény \\ \hline
	\rownum & 2021.09.08. & Céges kültéri esemény \\ \hline
	\rownum & 2021.09.10. & Git/Sourcetree beüzemelés \\ \hline
	\rownum & 2021.09.13. & Git/Sourcetree gyakorlás \\ \hline
	\rownum & 2021.09.13. & PADS tréning \\ \hline
	\rownum & 2021.09.16. & Board design frissítés (alkatrészek cseréje és séma frissítése) \\ \hline
	\rownum & 2021.09.16. & Board design frissítés \\ \hline
	\rownum & 2021.09.17. & Board design frissítés \\ \hline
	\rownum & 2021.09.17. & Board design frissítés \\ \hline
	\rownum & 2021.09.20. & Hatótáv mérés megtervezése \\ \hline
	\rownum & 2021.09.23. & Hatótáv mérés kikísérletezése \\ \hline
	\rownum & 2021.09.23. & Hatótáv mérése a Duna partján \\ \hline
	\rownum & 2021.09.24. & Hatótáv mérése a Duna partján \\ \hline
	\rownum & 2021.09.24. & Hatótáv mérése a Duna partján \\ \hline
	$\sum$ & 220 óra& \\ \hline
	\end{tabular}
\end{table}
\vspace{.5 cm} \\
Gyakorlóhelyi konzulens:

\end{document}