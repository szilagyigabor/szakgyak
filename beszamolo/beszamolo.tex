\documentclass[a4paper,12pt,titlepage]{article}

%\usepackage{ucs}
\usepackage[T1]{fontenc}
\usepackage[utf8]{inputenc}
\usepackage[magyar]{babel}
\usepackage{amsfonts}
\usepackage{amsmath}
\usepackage{amssymb}
\usepackage{graphicx}
\usepackage[hang]{caption}
%\usepackage{subcaption}
%\usepackage{epsfig}
\usepackage{enumerate}
%\usepackage{psfrag}
\usepackage[left=20mm,right=20mm,top=20mm,bottom=25mm]{geometry}
% \usepackage[left=10mm,right=10mm,top=10mm,bottom=15mm]{geometry} %landscape
\usepackage[hyphenbreaks]{breakurl}
\usepackage[hyphens]{url}
%\usepackage{multirow}
%\usepackage{booktabs}
%\usepackage{tikz}
\usepackage{hyperref}
\usepackage{listings}
%\usepackage{csquotes}
\usepackage{siunitx}
\usepackage{xcolor}

\hypersetup{
	colorlinks,
	linkcolor={red!50!black},
	citecolor={blue!50!black},
	urlcolor={blue!80!black}
}

% \newcommand{\zi}{z^{-1}}
% \newcommand{\zii}{z^{-2}}
% \newcommand{\ejth}{e^{-j\vartheta}}
% \newcommand{\ejtth}{e^{-j2\vartheta}}
% \newcommand{\Ohm}{\Omega}
% \newcommand{\jw}{j\omega}
% \newcommand{\jwn}{(\jw)^2}
\newcommand{\cpv}[1]{\overline{#1}}
\newcommand{\Ukcs}{\cpv{U}}
\newcommand{\Icpv}{\cpv{I}}
\newcommand{\Ucpv}{\cpv{U}}
%\usetikzlibrary{arrows.meta}

\pagestyle{plain} 

\definecolor{codegreen}{rgb}{0,0.6,0}
\definecolor{codegray}{rgb}{0.5,0.5,0.5}
\definecolor{codepurple}{rgb}{0.58,0,0.82}
\definecolor{backcolour}{rgb}{0.95,0.95,0.92}

\lstdefinestyle{mystyle}{
    backgroundcolor=\color{backcolour},   
    commentstyle=\color{codegreen},
    keywordstyle=\color{magenta},
    numberstyle=\tiny\color{codegray},
    stringstyle=\color{codepurple},
    basicstyle=\ttfamily\scriptsize,
    breakatwhitespace=false,         
    breaklines=true,                 
    captionpos=b,                    
    keepspaces=true,                 
    numbers=left,                    
    numbersep=5pt,                  
    showspaces=false,                
    showstringspaces=false,
    showtabs=false,                  
    tabsize=2
}

\lstset{style=mystyle}

\title{
\includegraphics[width=0.6\textwidth]{kep/bme_logo.pdf} \\
\centering
\large{\textbf{Budapesti Műszaki és Gazdaságtudományi Egyetem}\\
\textbf{Villamosmérnöki és Informatikai Kar}\\
\textbf{Szélessávú Hírközlés és Villamosságtan Tanszék}}\\
\vspace{6cm}
\huge{\textbf{MSc Szakmai Gyakorlat Beszámoló}} \\
\vspace{2cm}}
\author{Szilágyi Gábor \\\vspace{2cm}\\ Konzulens: Bódi Tamás}
\date{Budapest, \today}
  

\begin{document}
  	\maketitle
	\section{Bevezetés}
	A Silicon Laboratories Hungary kft-nél 2021 júliusában kezdtem el dolgozni gyakornokként és az elmúlt évben sokat tanultam a munkám során. Az MSc-s szakmai gyakorlat ledokumentált időtartama alatt már komolyabb feladataim voltak, mint a gyakornokoskodásom elején, ezeket az elvégzett feladatokat, a használt eszközöket és keretrendszert mutatom be ebben a beszámolóban. A cégen belül a ,,Hardware Tools" csapatban dolgoztam, ez a csapat elsősorban a rádiós demonstrációs áramkörök tervezéséért és az ezekhez kapcsolódó mérésekért felelős.
	\subsection{A cégről}
		A fent említett cég a Silicon Laboratories (SiLabs) austini központú (USA, Texas), multinacionális vállalat budapesti alvállalata. A cég rendelkezik irodákkal Austinon és Budaesten kívül többek között Hyderabadban (India) Osloban (Norvégia) és Rennes-ben (Franciaország). A világ több sarkában való működés ellenére összesen körülbelül 2000-en dolgoznak a cégnél, a budapesti iroda jelenlegi létszáma 200 körüli, ami a cégen belül az egyik legnépesebb helyszínnek számít. Az egyes helyszínek valamelyest specializálódtak a feladatkörükben, például Austinban történik maguknak a szilícium chipeknek a fejlesztése, Budapesten hardver validációval és szoftverfejlesztéssel foglalkoznak nagy számban, norvégiában pedig a demonstrációs áramkörök gyártása történik.
		\par
		A SiLabs profilja a rádiós vagy rádió nélküli mikrokontroller integrált áramkörök (IC-k) tervezése, eladása, valamint hardveres és szoftveres támogatás nyújtása ezekhez az ügyfeleknek. Az ügyfelek nem végfelhasználók, hanem más cégek, akik a SiLabs chipjeit és támogatását felhasználva készítenek termékeket. Ilyen ügyfél például az Ikea, az Amazon, a Google és az Apple. Az Ikea példájánál maradva, a projekt, amivel személyesen találkoztam és ehhez az ügyfélhez köthető, az okos villanykörtékbe szánt rádiós modulok. Egy másik egyszerűbb alkalmazása a SiLabs rádióinak a kapunyitó távirányító, amikkel gyakran találkozom a laborban a munkatársaimnál. A komplexebb alkalmazások közé tartoznak az okos otthon kialakításához használható eszközök. Ilyen az Amazon Alexa nevű terméke, amely az okos otthon rendszer egy központi, irányító eleme lehet, de ennek a rendszernek más elemei is használhatnak SiLabs chipeket, mint például az okos szenzorok vagy vezérlőeszközök.
		\par
		A Hardware Tools csapat által tervezett és tesztelt demonstrációs áramkörök referencia implementációként szolgálnak a kérdéses IC-ket használó eszközökhöz. Ezeknek az áramköröknek a rádiófrekvenciás része főleg nagyobb frekvenciákon (pl. 2,4 GHz) nagyon érzékeny a nyomtatott áramköri rajz mintázatára és a beültetett alkatrészek elhelyezkedésére. Egy jól működő implementációtól való kis mértékű eltérés is nagyban ronthatja az áramkör teljesítőképességét több szempontból is. Ilyenkor legtöbbször az fordul elő, hogy az alapharmonikus frekvencián, ahol a hasznos adatátvitel történik, lecsökken a kisugárzott teljesítmény és a vételi érzékenység. Egy másik tünete a rosszul megtervezett NYÁK-nak a felharmonikus frekvenciákon a megnövekedett kisgugárzott teljesítmény. Az első probléma miatt az eszköz nem tudja elég jól vagy egyáltalán ellátni a feladatát, a második probléma miatt pedig más eszközök működését akadályozhatja, mivel nem felel meg az elelktromágneses kompatibilitási (EMC) követelményeknek.
	\section{RF Mérés}
		\subsection{Vezetett mérés}
		\subsection{Sugárzott mérés}
		\subsection{Modulációs faktor}
	\section{NyHL tervezés}
		\subsection{Kapcsolási rajz}
		\subsection{Nyomtatott huzalozási rajz}
\end{document}
